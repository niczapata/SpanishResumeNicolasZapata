\documentclass[11pt, a4paper]{article}
\usepackage[utf8]{inputenc}
% \usepackage[english, spanish]{babel}
\usepackage[margin=1in]{geometry}
\usepackage{enumitem}
\usepackage{hyperref}
\usepackage{xcolor}
\usepackage{titlesec}
\usepackage{parskip}
\usepackage{setspace}

% ===== CONFIGURACIÓN DE COLORES =====
\definecolor{accent}{RGB}{40, 100, 200}
\hypersetup{
    colorlinks=true,
    linkcolor=accent,
    urlcolor=accent,
    citecolor=accent
}

% ===== FORMATO DE SECCIONES =====
\titleformat{\section}{\large\bfseries\color{accent}}{}{0em}{}[\titlerule]
\titlespacing*{\section}{0pt}{12pt}{6pt}

% ===== ENCABEZADO =====
\begin{document}
\begin{center}
	{\Huge \textbf{Nicolás Zapata Álzate}} \\[6pt]
	{\large \textbf{Ingeniero de Software | Especialista en Automatización \& Datos}} \\[12pt]
	% CONTACTO en texto plano (ATS-friendly)
	\begin{tabular}{@{}c@{}}
		Manizales, Colombia | (+57) 3128856774                                                              \\
		\href{mailto:nicozapata@msn.com}{nicozapata@msn.com}                                                \\
		Portfolio: \href{https://nicolas-zapata-portfolio.vercel.app/}{nicolas-zapata-portfolio.vercel.app} \\
		LinkedIn: \href{https://www.linkedin.com/in/nicolas-zapata-al/}{linkedin.com/in/nicolas-zapata-al}  \\
		GitHub: \href{https://github.com/niczapata}{github.com/niczapata}                                   \\
	\end{tabular}
\end{center}

% ===== PERFIL PROFESIONAL (SUMMARY) - TOTALMENTE REDISEÑADO =====
\section*{SUMMARY}
\begin{spacing}{1.1}
	Ingeniero de Software con especialización en datos y más de 3 años de experiencia diseñando \textbf{soluciones de automatización}, \textbf{pipelines de datos} e \textbf{integración de sistemas} usando Python y Node.js. Experto en construir APIs escalables (FastAPI), implementar infraestructura como código (Terraform, AWS) y aplicar prácticas de calidad (Unit Testing) para garantizar la robustez de los procesos automatizados. Busco el rol de \textbf{Ingeniero de Datos - Especialista en Automatización} para aportar en la creación de soluciones de alto valor.
\end{spacing}

% ===== IDIOMAS Y EDUCACIÓN =====
\subsection*{IDIOMAS}
\begin{itemize}[leftmargin=*, nosep]
	\item \textbf{Español:} Nativo
	\item \textbf{Inglés:} Nivel B2 (Intermedio Avanzado)
\end{itemize}

\subsection*{EDUCACIÓN}
\textbf{Universidad Autónoma de Manizales.} \\
\textbf{Diplomado de inglés.} \\
\textit{Febrero 2025 – Actualmente} \\ Enfocado en competencias de Cambridge English.

\textbf{Fundación Universitaria Internacional de La Rioja (UNIR).} \\
\textbf{Pregrado en Ingeniería Informática.} \\
\textit{Febrero 2019 – Julio 2023} \\ Énfasis en Ingeniería de Software y Sistemas.

% ===== EXPERIENCIA LABORAL - OPTIMIZADA CON PALABRAS CLAVE =====
\section*{EXPERIENCIA LABORAL}
\textbf{Software Developer – Grupo Quanam Colombia SAS} \hfill \textit{Nov 2023 – May 2024}
\begin{itemize}[leftmargin=*, nosep, topsep=4pt]
	\item \textbf{Automaticé flujos de trabajo clave} diseñando e implementando 5+ módulos centrales en Odoo 16 (Inventario, CRM) con \textbf{Python, XML y JavaScript}, \textbf{reduciendo el tiempo de procesamiento manual en un 30\%}.
	\item Lideré la \textbf{migración e integración} de sistemas de nómina y facturación electrónica (DIAN) mediante APIs, mejorando la precisión y el cumplimiento normativo.
	\item Desarrollé un \textbf{sistema de reportes automatizado} y paneles de control, proporcionando información en tiempo real para la toma de decisiones.
\end{itemize}

\textbf{Analista \& Desarrollador Full-Stack – Conviventia} \hfill \textit{Feb 2023 – Abr 2023}
\begin{itemize}[leftmargin=*, nosep, topsep=4pt]
	\item \textbf{Automaticé el reporte y categorización} de más de 500 participantes mediante un \textbf{pipeline de datos} interno construido con \textbf{Node.js, React y SQL}, \textbf{eliminando 15+ horas de trabajo manual mensual}.
	\item \textbf{Integré y centralicé} datos dispersos de múltiples fuentes (formularios JotForm, hojas de cálculo), asegurando su exactitud y accesibilidad para distintos departamentos.
\end{itemize}

\textbf{Desarrollador Móvil (Flutter) – Vibbo} \hfill \textit{Abr 2021 – May 2021}
\begin{itemize}[leftmargin=*, nosep, topsep=4pt]
	\item Contribuí al desarrollo de una aplicación móvil MVP, implementando la \textbf{sincronización de datos} a través de \textbf{APIs REST} con autenticación JWT y una base de datos \textbf{PostgreSQL}.
\end{itemize}

% ===== EXPERIENCIA EN INVESTIGACIÓN - ENFOCADA EN INGENIERÍA =====
\section*{INVESTIGACIÓN \& DESARROLLO}
\textbf{Investigador en IA / Ingeniero de ML – UNIR} \hfill \textit{Mar 2020 – Dic 2022}
\begin{itemize}[leftmargin=*, nosep, topsep=4pt]
	\item Diseñé e implementé \textbf{pipelines de machine learning} en Python (aprendizaje supervisado \& profundo) utilizando \textbf{scikit-learn, TensorFlow y Pandas} para procesar conjuntos de datos personalizados.
	\item Apliqué mejores prácticas de ingeniería de software para desarrollar código escalable y documentado para investigación algorítmica con aplicaciones en el mundo real.
\end{itemize}

% ===== PROYECTOS - SELECCIÓN ESTRATÉGICA =====
\section*{PROYECTOS DESTACADOS}
\textbf{Backend Template con FastAPI \& AWS Cognito} \hfill \textbf{Python, FastAPI, AWS (Lambda, DynamoDB), Docker, Terraform}
\begin{itemize}[leftmargin=*, nosep, topsep=4pt]
	\item Plantilla de backend segura y escalable que demuestra arquitectura moderna, \textbf{autenticación JWT con AWS Cognito}, gestión de usuarios y \textbf{Infraestructura como Código (IaC)}.
	\item \textit{Repositorio:} \href{https://github.com/niczapata/FastAPI\_AWS\_Cognito\_Template}{github.com/niczapata/FastAPI\_AWS\_Cognito\_Template}
\end{itemize}

\textbf{Modelo de Clasificación de Calidad de Café (Proyecto de Tesis)} \hfill \textbf{Python, TensorFlow, Visión por Computador}
\begin{itemize}[leftmargin=*, nosep, topsep=4pt]
	\item Proyecto completo de ML que \textbf{automatiza la clasificación} de granos de café. Incluyó la creación manual del dataset, entrenamiento del modelo y validación de resultados.
\end{itemize}

\textbf{Pipeline de Análisis \& Clasificación de Datos} \hfill \textbf{Python, scikit-learn, pandas, NumPy}
\begin{itemize}[leftmargin=*, nosep, topsep=4pt]
	\item \textbf{Pipeline de procesamiento} para dataset CSV que realiza limpieza, análisis de características y entrenamiento de un modelo de árbol de clasificación para generar insights.
	\item \textit{Repositorio:} \href{https://github.com/niczapata/Classification\_Tree\_UNIR}{github.com/niczapata/Classification\_Tree\_UNIR}
\end{itemize}

% ===== HABILIDADES TÉCNICAS - REORGANIZADAS PARA EL PUESTO =====
\section*{TECHNICAL SKILLS}
\begin{itemize}[nosep, leftmargin=*]
	\item \textbf{Automatización \& Backend:} Python, Node.js, FastAPI, \textbf{Unit Testing (Pytest)}, APIs REST (JSON, HTTP), OAuth 2.0 / JWT, Autenticación Segura.
	\item \textbf{Cloud \& DevOps (AWS):} Lambda, API Gateway, S3, IAM, CloudWatch, \textbf{Terraform (IaC)}, Docker, CI/CD, Microsoft Azure.
	\item \textbf{Herramientas de Automatización/RPA:} \textbf{n8n}, Make (Integromat), Zapier.
	\item \textbf{Data \& AI/ML:} Pipelines de datos (ETL), Pandas, NumPy, scikit-learn, SQL, PostgreSQL, MySQL, TensorFlow.
	\item \textbf{Frontend \& Movilidad:} React, React Native, JavaScript, Flutter, HTML/CSS.
\end{itemize}

% ===== HABILIDADES BLANDAS - NUEVA SECCIÓN =====
\section*{SOFT SKILLS}
Capacidad analítica para resolución de problemas, proactividad para identificar oportunidades de mejora, comunicación efectiva con equipos técnicos y no técnicos, trabajo colaborativo en entornos híbridos y adaptabilidad a cambios y nuevos requerimientos.

% ===== CERTIFICACIONES =====
\section*{CERTIFICACIONES}
\begin{itemize}[leftmargin=*, nosep]
	\item \textbf{Introduction to Microsoft Azure Cloud Services} – Febrero 2025 (Coursera)
	\item \textbf{Google AI Essentials} – Agosto 2024 (Credly)
	\item \textbf{Java Career} – Platzi (Java SE, Java EE)
	\item \textbf{Fundamentos de JavaScript, Git, Bases de Datos} – Platzi
\end{itemize}

\end{document}

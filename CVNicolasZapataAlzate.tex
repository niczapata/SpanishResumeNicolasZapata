\documentclass[11pt, a4paper]{article}
\usepackage[utf8]{inputenc}
\usepackage[english, spanish]{babel}
\usepackage[margin=1in]{geometry}
\usepackage{enumitem}
\usepackage{hyperref}
\usepackage{xcolor}
\usepackage{fontawesome5}
\usepackage{multicol}
\usepackage{titlesec}
\usepackage{parskip}
\usepackage{graphicx}
\usepackage{setspace}

% ===== CONFIGURACIÓN DE COLORES =====
\definecolor{accent}{RGB}{40, 100, 200}
\hypersetup{
    colorlinks=true,
    linkcolor=accent,
    urlcolor=accent,
    citecolor=accent
}

% ===== FORMATO DE SECCIONES =====
\titleformat{\section}{\large\bfseries\color{accent}}{}{0em}{}[\titlerule]
\titlespacing*{\section}{0pt}{12pt}{6pt}

% ===== ENCABEZADO =====
\begin{document}
\begin{center}
    {\Huge \textbf{Nicolás Zapata Álzate}} \\[6pt]
    {\large \textbf{Ingeniero informático | Desarrollador de software}} \\[2pt]
    % {\footnotesize 12/03/1999} \\[12pt]
\end{center}

% ===== OBJETIVO =====
\section*{PERFIL}
Especialista en desarrollo de software con enfoque en la optimización de procesos digitales y automatización. Experiencia en el diseño de APIs, microservicios y bases de datos SQL y NoSQL, con conocimientos en seguridad informática. Me interesa colaborar con empresas que apuesten por la innovación y la mejora continua en tecnología. Mi objetivo es crear soluciones de software para mejorar y hacer más fácil el uso cotidiano, y contribuir a la industria del software.
% ===== COLUMNAS: CONTACTO, EDUCACIÓN, IDIOMAS =====
\begin{multicols}{2}

  \subsection*{CONTACTO}

  \begin{itemize}[leftmargin=*, nosep, label={}]
      \item \faGlobeAmericas\ Manizales, Caldas, Colombia.
      \item \faPhone\ (+57) 3128856774
      \item \faEnvelope\ \href{mailto:works\_nicolasz@outlook.com}{works\_nicolasz@outlook.com}
      \item \faGlobe\ \href{https://nicolas-zapata-portfolio.vercel.app/}{nicolas-zapata-portfolio.vercel.app}
      \item \faLinkedin\ \href{https://www.linkedin.com/in/nicolas-zapata-al/}{linkedin.com/in/nicolas-zapata-al}
      \item \faGithub\ \href{https://github.com/niczapata}{github.com/niczapata}
      \item \faGitlab\ \href{https://gitlab.com/NicolasZapata}{gitlab.com/NicolasZapata}
      \item \faTwitter\ \href{https://twitter.com/niczapata12}{twitter.com/niczapata12}
  \end{itemize}

  \subsection*{IDIOMAS}
  \begin{itemize}[leftmargin=*, nosep]
      \item \textbf{Español:} Nativo
      \item \textbf{Inglés:} B2
  \end{itemize}

  \columnbreak

  \subsection*{EDUCACIÓN}

  \textbf{Universidad Autónoma de Manizales.} \\
  \textbf{Diplomado de inglés.} \\
  \textit{Conclusión: En curso actualmente.}

  \textbf{Fundación Universitaria Internacional de La Rioja (UNIR).} \\
  \textbf{Pregrado en Ingeniería Informática.} \\
  \textit{Conclusión: 08/06/2023.}

\end{multicols}

% ===== EXPERIENCIA LABORAL =====
\section*{EXPERIENCIA}
\textbf{Desarrollador – Grupo Quanam Colombia SAS} \hfill \textit{Noviembre 2023 – Mayo 2024} \\
\begin{itemize}[leftmargin=*, nosep, topsep=4pt]
    \item Desarrollo de Módulos para Soluciones Empresariales con Odoo16 utilizando Python, XML y Javascript.
    \item \textbf{Migración de Nómina:} Actualización de módulos de nómina (gestión de vacaciones, contratos, procesamiento, préstamos, novedades, afiliaciones) de Odoo 13/14 a Odoo 16.
    \item \textbf{Módulo de caja menor:} Diseño e implementación de un sistema robusto para seguimiento de gastos, reembolsos y reportes, reduciendo errores manuales.
    \item \textbf{Módulo de Código de Barras para Envíos:} Desarrollo de un sistema integrado con inventario y logística para facilitar el manejo de productos.
    \item \textbf{Recepción de Pesaje para Cliente Agrícola:} Solución automatizada para medición de peso y captura de datos, mejorando la visibilidad de la cadena de suministro.
    \item \textbf{Sistema de informes:} Creación de herramientas y paneles de control dinámicos para varias empresas, proporcionando información en tiempo real.
    \item Soporte auxiliar para módulo de facturación y nómina electrónica para la DIAN mediante un Proveedor de API (Dataico).
\end{itemize}

\textbf{Auxiliar de TIC – Conviventia} \hfill \textit{Febrero 2023 – Abril 2023} \\
\begin{itemize}[leftmargin=*, nosep, topsep=4pt]
    \item Lideré la creación de un panel integral en Google Sheets (con datos de JotForm) para categorizar participantes (Practicantes, Voluntarios, Extranjeros) de la temporada 2023.
    \item Identifiqué áreas de automatización dentro de la fundación y propuse mejoras para agilizar operaciones.
\end{itemize}

\textbf{Mobile Developer – Vibbo} \hfill \textit{Abril 2021 – Mayo 2021} \\
\begin{itemize}[leftmargin=*, nosep, topsep=4pt]
    \item Contribuí al desarrollo de una aplicación móvil con Flutter.
    \item Sincronicé datos mediante API REST desde una base de datos PostgreSQL con autenticación en FastAPI.
    \item Integré fase inicial de función de videoconferencia y autenticación de usuarios con Firebase.
\end{itemize}

\textbf{Investigador en Inteligencia Artificial – UNIR} \hfill \textit{Marzo 2020 – Diciembre 2022} \\
\begin{itemize}[leftmargin=*, nosep, topsep=4pt]
    \item Investigación en modelos de Machine Learning, Deep Learning y Aprendizaje Supervisado.
    \item Proyectos: sensor para cuidado de tilapias, robots con detección de imágenes, y algoritmos estadísticos para estrategias de venta.
\end{itemize}

\textbf{Mobile Developer – Digital Brainly Solutions} \hfill \textit{Enero 2021 – Abril 2021} \\
\begin{itemize}[leftmargin=*, nosep, topsep=4pt]
    \item Desarrollo de la plantilla del chat para la aplicación Carry-App con React Native.
    \item Implementación del Front-end del chat para un proyecto de taxis.
    \item Creación de una Landing Page utilizando Ionic y Cordova.
\end{itemize}

% ===== CONOCIMIENTOS =====
\section*{CONOCIMIENTOS}
\begin{itemize}[leftmargin=*, nosep, topsep=4pt]
    \item \textbf{Lenguajes de Programación:} Java, Kotlin, C, C++, C\#, JavaScript, Python, Lua, SQL, Dart, HTML, CSS, R, LaTeX.
    \item \textbf{Herramientas de desarrollo:} Cline, Android Studio Arduino, Unity, Windsurf, Cursor, Visual Studio Code, IntelliJ IDEA, PyCharm, NeoVim, RStudio.
    \item \textbf{Frameworks y librerías:} React, Odoo, Vite, Next.js, Node.js, Express.
    \item \textbf{Bases de datos:} MySQL, PostgreSQL, SQLite, MongoDB, Firebase.
    \item \textbf{Desarrollo Móvil:} React Native, Flutter, Ionic, Android Studio.
    \item \textbf{Otras Capacidades:} Notion, Trello, Jira, Figma, Git, GitHub, Docker, GitLab, BitBucket, Microservicios, Azure, ChatGPT, DeepSeek, Prompt Engineering, Antropic Claude.
    \item \textbf{Sistemas Operativos:} Arch Linux, Ubuntu, Linux Mint, Manjaro, Kali Linux.
    \item \textbf{Inteligencia Artificial:} Pandas, Numpy, Scikit-learn, Tensorflow, PyTorch, OpenCV, Keras, Matplotlib.
\end{itemize}

% ===== APTITUDES Y HABILIDADES =====
\section*{APTITUDES Y HABILIDADES}
\begin{itemize}[leftmargin=*, nosep, topsep=4pt]
    \item Puntual y comprometido con los plazos de trabajo.
    \item Pensamiento recursivo para encontrar soluciones alternativas a problemas.
    \item Flexibilidad para adaptar estrategias de desarrollo cuando es necesario.
    \item Innovación constante e integración de nuevas tecnologías y estándares.
    \item Pasión por probar nuevas tecnologías y trabajar con herramientas emergentes.
    \item Alta adaptabilidad a entornos Windows y Linux, y a diferentes metodologías de software.
    \item Productividad mejorada mediante el uso de herramientas como Vim y asistentes de IA (Codeium, Continue) para autocompletado de código.
\end{itemize}

% ===== CURSOS =====
\section*{CURSOS}
\begin{itemize}[leftmargin=*, nosep]
    \item Google AI Essentials – Coursera
    \item Introduction to Microsoft Azure Cloud Services – Coursera
    \item Semillero de Machine Learning – UNIR
    \item Semillero Quinoa – UNIR
    \item Fundamentos de Soporte IT – LinkedIn
    \item Carrera de Java – Platzi
    \item Curso fundamentos de JavaScript – Platzi
\end{itemize}

% ===== PROYECTOS / PORTAFOLIO =====
\section*{PROYECTOS / PORTAFOLIO}
\textbf{Algoritmo de clasificación de imágenes en granos de café} \\
Modelo entrenado con Teachable Machine para seleccionar frutos de café rojos de los verdes, basado en muestras recolectadas.

\textbf{CatBreed App} \\
App desarrollada con React Native que conecta con una API para desplegar una lista de gatos con imagen, nombre, raza y descripción. \\
\textit{Repositorio:} \href{https://github.com/niczapata/CatBreedApp}{github.com/niczapata/CatBreedApp}

\textbf{Todo React Vite} \\
Aplicación de lista de tareas desarrollada con React y Vite. \\
\textit{Repositorio:} \href{https://github.com/NicolasZapata/TodoReactVite}{github.com/NicolasZapata/TodoReactVite}

\textbf{Jetpack Compose Application} \\
Aplicación móvil nativa para Android que demuestra la metodología Jetpack Compose en Kotlin. \\
\textit{Repositorio:} \href{https://github.com/NicolasZapata/Jetpack-Compose-Application}{github.com/NicolasZapata/Jetpack-Compose-Application}

\textbf{Fazt Web's Node.js Simple Page} \\
Página web desarrollada con Node.js. \\
\textit{Repositorio:} \href{https://github.com/NicolasZapata/FastWeb-Nodejs-Practice}{github.com/NicolasZapata/FastWeb-Nodejs-Practice}

\textbf{Classification Tree UNIR} \\
Árbol de clasificación realizado en Python con Sklearn, pandas, numpy y statmodels.api sobre datos extraídos de un archivo CSV. \\
\textit{Repositorio:} \href{https://github.com/NicolasZapata/Classification\_Tree\_UNIR}{github.com/NicolasZapata/Classification\_Tree\_UNIR}

\textbf{OpenCV – UNIR} \\
Laboratorio de manipulación de imágenes con Python y OpenCV. \\
\textit{Repositorio:} \href{https://github.com/NicolasZapata/OpenCV-Unir}{github.com/NicolasZapata/OpenCV-Unir}

\textbf{Data Mining Unir} \\
Tratamiento de datos con Python y DTale. \\
\textit{Repositorio:} \href{https://github.com/NicolasZapata/Data\_Mining\_Unir-1}{github.com/NicolasZapata/Data\_Mining\_Unir-1}

\textbf{Fashion Deep Learning | Semillero Quinoa | UNIR} \\
Entrenamiento de un dataset de ropa con TensorFlow, Keras, NumPy y Matplotlib, utilizando valores flotantes. \\
\textit{Repositorio:} \href{https://github.com/NicolasZapata/Fashion-Deep-Learning--Semillero-Quinoa}{github.com/NicolasZapata/Fashion-Deep-Learning}

% ===== REFERENCIAS PERSONALES =====
\section*{REFERENCIAS PERSONALES}
\begin{itemize}[leftmargin=*, nosep]
    \item \textbf{Jhon Jairo Grisales Becerra} \\ Coordinador técnico – ARAGON CONSTRUCCIONES Y MINERIA \\ Teléfono: +57 300 372 2372
    \item \textbf{Viviana María Zapata Álzate} \\ Profesional Social – INTERPARQUE TECNOLOGICO \\ Teléfono: +57 312 896 5368
    \item \textbf{Leonardo Garavito} \\ Desarrollador de Negocios y CEO – GRUPO QUANAM COLOMBIA \\ Teléfono: +57 311 2308556
    \item \textbf{Jhon Carlos Colorado Angulo} \\ Desarrollador Front-End – ARIADNA COMMUNICATIONS GROUP \\ Teléfono: +57 3205336397
\end{itemize}

% ===== ESPACIO PARA FOTO (OPCIONAL) =====
% Descomenta las siguientes líneas y agrega tu imagen si lo deseas:
% \begin{figure}[h]
%     \centering
%     \includegraphics[width=3cm]{tu_foto.jpg}
%     \label{fig:foto}
% \end{figure}

\end{document}
